%D \module
%D   [      file=p-dabeer,
%D        version=0.05,
%D          title=\CONTEXT\ User Module,
%D       subtitle=dabeer for Persian typesetting,
%D         author={Mohammad Hossein Bateni},
%D           date=\currentdate,
%D      copyright={Mohammad Hossein Bateni},
%D        license=GNU General Public License v2]

% dabeer module for Persian typesetting in ConTeXt
% Copyright (C) 2015, 2016  Mohammad Hossein Bateni

% This program is free software; you can redistribute it and/or
% modify it under the terms of the GNU General Public License
% as published by the Free Software Foundation; either version 2
% of the License, or (at your option) any later version.

% This program is distributed in the hope that it will be useful,
% but WITHOUT ANY WARRANTY; without even the implied warranty of
% MERCHANTABILITY or FITNESS FOR A PARTICULAR PURPOSE.  See the
% GNU General Public License for more details.

% You should have received a copy of the GNU General Public License
% along with this program; if not, write to the Free Software
% Foundation, Inc., 59 Temple Place - Suite 330, Boston, MA  02111-1307, USA.

\setgvalue{DabeerVersion}{0.05}
%% TODO: add automatic conversion from Latin to Persian numbers in Lua.
\setgvalue{DabeerVersionFa}{\LTR{۰٫۰۵}}
\writestatus{dabeer}{Loading module version \DabeerVersion}

%D It is sometimes easier to use Persian percent sign for commenting.
\catcode`٪=14

%% TODO: method should be an option
%D Set up the default directions.

%D \type{method=default} is not recommended at all.  \type{method=one}
%D and \type{method=two} are both good.
\setupdirections[bidi=global,method=one]
\setupalign[r2l]

%D Convenience macros for text direction.  These give \type{\LTR} and
%D \type{\RTL} in addition to their start/stop block environments.
\definestartstop[LTR][before={\begingroup\lefttoright},after=\endgroup]
\definestartstop[RTL][before={\begingroup\righttoleft},after=\endgroup]

\writestatus{dabeer}{Setting up main document fonts}
%% TODO: These should receive font names via options.
%D Basic font definitions:
%D Previously, in order to avoid typescripts, we needed to use the
%D module \type{simplefonts} and macros like \type{\setmainfont}
%D defined therein.
%D For some reason, the following does not work properly if we simply
%D provide the font's name: In case of my Farsi fonts, the default
%D style will then be italic.
%D The additional benefit, however, is that we can map the italic style
%D to oblique, which better suits Persian text.  However, neither italic
%D nor oblique look good in Persian text.
\definefontfeature[tlig][default][tlig=yes]  % for en- and em-dashes
\definefontfeature
   [farsihm]    
   [mode=node,language=far,script=arab,
    init=yes,medi=yes,fina=yes,isol=yes,
    liga=yes,dlig=yes,rlig=yes,clig=yes,
    mark=yes,mkmk=yes,kern=yes,curs=yes
   ]
\definefontfamily [farsifont] [serif] []
                  [features={farsihm,tlig},
                    tf=file:HM_XNiloofar.ttf,
                    it=file:HM_XNiloofarOb.ttf,
                    bf=file:HM_XNiloofarBd.ttf,
                    bi=file:HM_XNiloofarObBd.ttf
                  ]
\definefontfamily [farsifont] [sans] []
                  [features={farsihm,tlig},
                    tf=file:HM_FMitra.ttf,
                    it=file:HM_FMitraOb.ttf,
                    bf=file:HM_FMitraBd.ttf,
                    bi=file:HM_FMitraObBd.ttf
                  ]
%D TeX ligatures are not desirable in mono fonts.
\definefontfamily [farsifont] [mono] [ALM Fixed]         [features=arabic]
\definefontfamily [farsifont] [math] [Latin Modern Math] [default]
\setupbodyfont    [farsifont]

%D We do not like to see special characters such as ZWNJ.
\setcharacterstripping[1]

%D Persian fonts usually have incorrect heights.  We fix this issue here.
\setupinterlinespace[line=3.2ex]  % default is 2.8ex

%D While we're at it, we might as well increase the parskip to make it more pleasing.
%D A more systematic way to handle this might be to use \type{\definewhitespacemethod}.
%D \type{halfline} and a bunch of other methods are already defined.
\definemeasure
  [defaultparskip]
  [0.3\baselineskip]

\setupwhitespace
  [\measure{defaultparskip}]

%D The following will fix \type{\quote} and \type{quotation}
%D delimiters.  I have not seen the half-guillemots in any real text
%D but I put it here as reference.  This is easy to change.
%% TODO: The following does not work in English texts.  So perhaps I
%% should put this under language customizations.
\def\guilsinglleft{‹}
\def\guilsinglright{›}
\setupquotation[left=\leftguillemot,right=\rightguillemot]
\setupquote[left=\guilsinglleft,right=\guilsinglright]

%D In Persian documents, we prefer all the numbers to use Persian
%D digits unless there is a reason not to (for instance, it is part of
%D an English text).  \CONTEXT\ provides several number conversions,
%D we plan to add a few more.  

%D We like to have Persian decimals for page numbers.  The proper place
%D to define these is in a style file (say, for an article or a book)
%D where, for instance, different numbering (digits, alphabetic, etc.)
%D may be used in frontmatter, body, appendices, etc.  Nonetheless, we
%D provide a default that works for most documents.  Same goes with
%D other numbering elements.
%% TODO: Is it possible to define ``shortcuts'' like n,g,a?
\setuppagenumber[numberconversion=persiandecimals]

