%% source: Farsi Wikipedia
خواجه شمس‌الدین محمد بن بهاءالدّین حافظ شیرازی 
(زاده ح.~۷۲۷ (قمری) -- درگذشته ۷۹۲ (قمری))،
معروف به {\em لسان‌الغیب}، {\em ترجمان الاسرار}، {\em لسان‌العرفا}
و {\em ناظم‌الاولیا} شاعر بزرگ سدهٔ هشتم ایران
(برابر قرن چهاردهم میلادی) و یکی از سخنوران نامی جهان است.
بیش‌تر شعرهای او غزل هستند که به غزلیات حافظ شهرت دارند.
گرایش حافظ به شیوهٔ سخن‌پردازی خواجوی کرمانی و
شباهت شیوهٔ سخنش با او مشهور است.
او از مهم‌ترین تأثیرگذاران بر شاعران پس از خود شناخته می‌شود.
در قرون هجدهم و نوزدهم اشعار او به زبان‌های اروپایی ترجمه شد
و نام او به گونه‌ای به محافل ادبی جهان غرب نیز راه یافت.
هرساله در تاریخ ۲۰ مهرماه مراسم بزرگ‌داشت حافظ در محل آرامگاه او در شیراز
با حضور پژوهشگران ایرانی و خارجی برگزار می‌شود.
مطابق تقویم رسمی ایران این روز روز بزرگ‌داشت حافظ نامیده شده است.

اطلاعات چندانی از خانواده و اجداد خواجه حافظ در دست نیست
و ظاهراً پدرش بهاءالدین نام داشته و مادرش نیز اهل کازرون بوده‌است.
در اشعار او که می‌تواند یگانه منبع موثّق زندگی او باشد
اشارات اندکی از زندگی شخصی و خصوصی او یافت می‌شود.
آنچه از فحوای تذکره‌ها به دست می‌آید بیشتر افسانه‌هایی است
که از این شخصیّت در ذهن عوام ساخته و پرداخته شده است.
با این همه آنچه با تکیه به اشارات دیوان او و برخی منابع معتبر
قابل بیان است آن است که او در خانواده‌ای از نظر مالی در حد متوسط
جامعه زمان خویش متولد شده است.
(با این حساب که کسب علم و دانش در آن زمان اصولاً
مربوط به خانواده‌های مرفه و بعضاً متوسط جامعه بوده است).
در نوجوانی قرآن را با چهارده روایت آن از بر کرده
و از همین رو به حافظ ملقب گشته است.

آرامگاه حافظ در شهر شیراز و در منطقهٔ حافظیّه در فضایی
آکنده از عطر و زیبایی گل‌های جان‌پرور، درآمیخته با شور
اشعار خواجه، واقع شده است. امروزه این مکان یکی از جاذبه‌های
مهمّ گردشگری به شمار می‌رود و بسیاری از مشتاقان شعر و اندیشه حافظ را
از اطراف جهان به این مکان می‌کشاند.

در زبان اغلب مردم ایران، رفتن به حافظیّه معادل با
زیارت آرامگاه حافظ گردیده است. اصطلاح زیارت که بیشتر
برای اماکن مقدّسی نظیر کعبه و بارگاه امامان به‌کار می‌رود،
به‌خوبی نشان‌گر آن است که حافظ چه چهرهٔ مقدّسی نزد ایرانیان دارد.
برخی از معتقدان به آیین‌های مذهبی و اسلامی، رفتن به آرامگاه او را
با آداب و رسوم مذهبی همراه می‌کنند. از جمله با وضو به آنجا می‌روند
و در کنار آرامگاه حافظ به نشان احترام، کفش خود را از پای بیرون می‌آورند.
سایر دل‌باختگان حافظ نیز به این مکان به‌عنوان سمبلی از عشق راستین
و نمادی از رندی عارفانه با دیدهٔ احترام می‌نگرند.
