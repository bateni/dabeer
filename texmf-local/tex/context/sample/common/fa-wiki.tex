%% source: Farsi Wikipedia
ویکی‌پدیا دانشنامه‌ای اینترنتی با بیش از ۲۸۰ زبان با محتوای آزاد است که با همکاری افراد داوطلب نوشته می‌شود و هر کس که به اینترنت دسترسی داشته باشد می‌تواند مقاله‌های آن را ویرایش کند.
هدف ویکی‌پدیا آفرینش و انتشار جهانی یک دانشنامهٔ آزاد به تمامی زبان‌های زندهٔ دنیاست. با رشد روزافزون این دانشنامه گردانندگان آن در بنیاد ویکی‌مدیا چندین پروژهٔ مشابه دیگر همچون ویکی‌واژه، ویکی‌کتاب، ویکی‌گفتاورد، ویکی‌خبر، ویکی‌دانشگاه، ویکی‌سفر، ویکی‌داده و ویکی‌گونه را پدید آوردند.

ویکی‌پدیای فارسی پس از دو سال از شروع پروژهٔ ویکی‌پدیای انگلیسی، در ۲۸ آذر ۱۳۸۲ (۱۹ دسامبر ۲۰۰۳) فعالیت خود را آغاز کرد و با تلاش و کوشش شبانه‌روزی کاربران، اکنون در ردهٔ هجدهم ویکی‌پدیاها قرار دارد و بزرگترین ویکی‌پدیا در میان زبان‌های خاورمیانه و زبان‌های راست به چپ و بزرگترین دانشنامهٔ فارسی محسوب می‌شود. ویکی‌پدیای فارسی هم اکنون (۱۶ دی ۱۳۹۴ خورشیدی) ۴۷۹٬۰۸۱ نوشتار دارد.
