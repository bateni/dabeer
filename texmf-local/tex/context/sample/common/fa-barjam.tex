%% source: Farsi Wikipedia

توافق جامع و نهایی هسته‌ای وین با عنوان شناخته شده و رسمی برنامه جامع اقدام مشترک یا برجام
\footnote{به انگلیسی: Joint Comprehensive Plan of Action}
 در راستای توافق جامع بر سر برنامه هسته‌ای ایران و بدنبال تفاهم هسته‌ای لوزان، در سه‌شنبه ۲۳ تیر ۱۳۹۴ (۱۴ ژوئیه ۲۰۱۵) در وین اتریش بین ایران، اتحادیه اروپا و گروه ۱+۵ (شامل چین، فرانسه، روسیه، پادشاهی متحد بریتانیا، ایالات متحده آمریکا و آلمان) منعقد شد.

مذاکرات رسمی برای طرح جامع اقدام مشترک دربارهٔ برنامه اتمی ایران با پذیرفتن توافق موقت ژنو بر روی برنامه هسته‌ای ایران در نوامبر ۲۰۱۳ شروع شد. به مدت ۲۰ ماه کشورها درگیر مذاکره بودند و در آوریل ۲۰۱۵ تفاهم هسته‌ای لوزان شکل گرفت.

تحت این توافق، ایران ذخایر اورانیم غنی شده متوسط خودش را پاکسازی خواهد کرد و ذخیره‌سازی اورانیوم با غنای کم را تا ۹۸ درصد قطع می‌کند، و تعداد سانتریفیوژها را حدود دوسوم و حداقل به مدت ۱۵ سال کاهش می‌دهد. ۱۵ سال بعد، ایران موافقت کرده است که اورانیوم را بیش از ۳٫۶۷درصد غنی‌سازی نکند یا تأسیسات غنی سازی اورانیوم جدید یا رآکتور آب‌سنگین جدیدی را نسازد. فعالیت‌های غنی سازی اورانیوم به مدت ۱۰ سال به یک تک ساختمان که از سنتریفیوج‌های نسل اول استفاده می‌کند محدود خواهد بود. دیگر تأسیسات نیز طبق پروتکل الحاقی آژانس بین‌المللی انرژی اتمی برای اجتناب از خطرهای تکثیر سلاح‌های اتمی تبدیل خواهند شد. برای نظارت و تأیید اجرای توافق نامه توسط ایران، آژانس بین‌المللی انرژی اتمی (IAEA) به تمام تأسیسات اتمی ایران دسترسی منظم خواهد داشت. در نتیجه این توافقنامه که تاییدیه پایدار متعهدین آن را به همراه دارد، ایران از تحریم‌های شورای امنیت ملل متحد، اتحادیه اروپا و ایالات متحده بیرون خواهد آمد.

این دور از مذاکرات برنامه هسته‌ای ایران و گروه ۵+۱ از ظهر روز شنبه ۲۷ ژوئن ۲۰۱۵، در هتل کوبورگ در شهر وین اتریش با حضور جان کری، وزیر امور خارجه ایالات متحده آمریکا و محمدجواد ظریف، وزیر امور خارجه ایران و هیئت‌های کارشناسی دو طرف آغاز شد. ضرب الاجل تعیین‌شده زمان این مذاکرات، روز سی‌ام ژوئن تعیین شده بود که سه‌بار به ترتیب تا سوم، دهم و سیزدهم ژوئیه تمدید گردید.


%%% According to http://tex.stackexchange.com/questions/26898/center-context-table-horizontally
%%% the \midaligned solution should work with \starttable and \startTABLE.
\startplacetable[location=force,number=no]
\righttoleft
%\midaligned{ % this gives an error
\starttabulate[format={|c|c|c|},align=middle]
\HL
\NC\bf
گنجایش
\NC\bf
پیش از برجام
\NC\bf
پس از برجام
\NC
\NR
\HL
\NC
گریزانه‌های نصب‌شده از نوع نسل نخست
\NC
۱۹٬۱۳۸
\NC
۶٬۱۰۴
\NC
\NR
\NC
گریزانه‌های نصب شده از نوع پیشرفته
\NC
۱٬۰۳۴
\NC
۰
\NC
\NR
\NC
زمان مورد نیاز ایران برای فرار از توافق
\NC
۱-۲ ماه
\NC
۱ سال
\NC
\NR
\NC
بهره‌گیری از گریزانه R\&P
\NC
غیراجباری
\NC
اجباری
\NC
\NR
\NC
ذخایر اورانیوم غنی‌شده با غلظت پایین
\NC
۱۹٬۲۱۱ پوند
\NC
۶۴۰ پوند
\NC
\NR
\NC
ذخایر اورانیوم غنی‌شده با غلظت میانه
\NC
۴۳۰ پوند
\NC
۰ پوند
\NC
\NR
\HL
\stoptabulate
%}
\stopplacetable
