%% source: Farsi Wikipedia
گوگل
\footnote{به انگلیسی: Google}
 یک شرکت سهامی عام است که در زمینهٔ جستجوی اینترنتی،
رایانش ابری و تبلیغات سرمایه‌گذاری می‌کند.
این شرکت توسط لری پیج و سرگئی برین تأسیس شد
که هر دوی آن‌ها در دانشگاه استنفورد به عنوان دانشجوی دکترا حضور داشتند
و با نام «مردان گوگل» شناخته می‌شدند.
گوگل در ابتدا به عنوان یک شرکت خصوصی در ۴ سپتامبر ۱۹۹۸ ثبت شد
 و فروش اولیه سهام آن در ۱۹ اوت ۲۰۰۴ انجام شد.
لری پیج و سرگئی برین و اریک اشمیت قبول کردند که به مدت بیست سال، یعنی تا سال ۲۰۲۴ در کنار هم کار کنند.
هدف گوگل از آغاز «سازماندهی کردن اطلاعات دنیا و دسترس‌پذیر کردن آن‌ها برای عموم» بود،
و شعار غیررسمی شرکت (که توسط مهندس گوگل امیت پاتل ابداع شد
و توسط پل بوچهیت از آن حمایت شد) «شرور نباشید» (به انگلیسی: Don't Be Evil) بود.
در سال ۲۰۰۶ شرکت به محل جدید و کنونیش در مانتین ویو، کالیفرنیا منتقل شد.

گوگل بر گرفته شده از واژهٔ گوگل
\footnote{Googol}
که به معنی «یک عدد یک و صد صفر جلوی آن» است که نوعی شعار و در واقع مقصود موضوع است.
بدین معنی که گوگل قصد دارد تا سرویس‌ها، اهداف و اطلاع‌رسانی و اطلاعات خود را تا آن مقدار در وب در جهان گسترش دهد.
گوگل به طور تخمینی دارای بیش از یک میلیون سرور در سراسر جهان است،
و روزانه برروی بیش از یک میلیارد درخواست جستجو،
و حدود ۲۴ پتابایت داده تولید شده توسط کاربران پردازش انجام می‌دهد.
نتیجه رشد سریع گوگل از بدو ثبت شدن شرکت، زنجیره‌ای از محصولات، خرید و شراکت با شرکت‌های دیگر بود.
این شرکت محصولات اینترنتی سودمندی مانند سرویس ایمیل جی‌میل و شبکه‌های اجتماعی گوگل+
را به کاربران خود پیشنهاد می‌دهد.
محصولات گوگل به ابزارهای رومیزی نیز توسعه پیدا کرد،
با برنامه‌هایی مانند مرورگر وب گوگل کروم، سازماندهی و ویرایشگر تصاویر پیکاسا و پیام‌رسان فوری گوگل تاک.
گوگل سازنده پرکاربرترین سیستم‌عامل موبایل جهان، اندروید و سیستم‌عامل گوگل کروم نیز هست.

الکسا اینترنت وبگاه گوگل در آمریکا (google.com)
 را به عنوان پر\-بازدید\-کننده‌\-ترین سایت اینترنت در فهرستش قرار داده است،
همچنین وبگاه‌های گوگل در کشورهای دیگر مثل هند google.co.in (که در رتبه چهاردهم قرار دارد)
به عنوان پر\-بازدید\-کننده‌\-ترین سایت در هند و یا google.co.uk
به عنوان پر\-بازدید\-کننده‌\-ترین سایت در انگلیس و خیلی سایت‌های دیگر آن پر\-بازدید\-کننده‌\-ترین در کشور خود هستند
و در فهرست برترین صد سایت دنیا قرار دارند.
سایر متعلقات گوگل مانند یوتیوب (رتبه سوم در الکسا) یا بلاگر (رتبه ششم در الکسا) و اورکات نیز در این فهرست قرار دارند.
گوگل همچنین توسط برندز به عنوان دومین برند باارزش دنیا شناخته شده است.
شکل تجارت کنونی گوگل و خدماتی که ارائه می‌دهد باعث انتقاد از این شرکت در موضوعاتی مانند حریم، حق تکثیر، و سانسور شده است.

مقامات گوگل ارتباط بسیار نزدیکی با کاخ سفید دارند.
