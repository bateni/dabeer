%% source: Farsi Wikipedia
تاریخ ژاپن شامل تاریخ رشته جزیره‌های ژاپن و مردم ژاپن است و از دوران پیش از تاریخ، تا تاریخ نوین این کشور به عنوان یک دولت-ملت را در بر می‌گیرد. نخستین ارجاع شناخته‌شده نوشتاری به ژاپن در قرن اول میلادی و در مجموعه متن‌های چینی بیست و چهار تاریخ دیده می‌شود. با این حال شواهد حاکی از آن است که بشر از اواخر دوران پارینه‌سنگی در جزیره‌های ژاپن زندگی می‌کرده است. الگوهای باثبات زندگی و نشانه‌های تمدن از پایان عصر یخبندان تا هزاره نخست پیش از میلاد ظاهر شد و از حدود پانصد سال پیش از میلاد در زمانی که به دوره یایویی مشهور است، کار با آهن‌آلات آغاز شد.

از حدود قرن سوم میلادی عصر تاریخ باستان این سرزمین آغاز می‌شود. فرهنگی که در این دوره شکل می‌گیرد، مبتنی بر مذهب شینتو است. در دوره آسوکا خط چینی برای نگارش زبان ژاپنی برگزیده شده و آیین بودایی از طریق سرزمین کره معرفی می‌شود. دوره هی‌آن عصر ظهور خاندان‌های پرقدرتی همچون خاندان فوجی‌وارا است که مترادف بود با پایان تاریخ باستان و آغاز دوران فئودالی در تاریخ ژاپن. این عصر فئودالی بعدها با ورود کشتی‌های جنگی امریکا به خلیج توکیو در سال ۱۸۵۳ میلادی، به پایان رسید.

در دوران نوین ژاپن به سرعت راه نوسازی را در پیش گرفت و به قدرتی جهانی مبدل شد. در جنگ جهانی اول ژاپن در کنار بریتانیا، فرانسه و روسیه قرار گرفت و در شورای جامعه ملل صاحب کرسی دائم شد. با این حال بعدها پس از یورش به کشور چین از این سازمان خارج شد.

بعد از آغاز جنگ جهانی دوم، در پی تحریم‌های نفتی امریکا، ژاپن به پرل هاربر حمله کرد. در نتیجه، امریکا وارد جنگ شد و با بمباران اتمی هیروشیما و ناگاساکی در سال ۱۹۴۵ میلادی ژاپن را به تسلیم واداشت. با پایان جنگ و اشغال هفت ساله ژاپن، این کشور به اصلاحات گسترده دموکراتیک و بازسازی نهادهای اقتصادی دست یازید. با امضای پیمان صلح سان‌فرانسیسکو ژاپن استقلال خود را دوباره به دست آورد. ژاپن در این دوره رشد سریعی را تجربه کرد و به یک قدرت بزرگ اقتصادی تبدیل شد. این کشور هم‌اکنون سومین قدرت اقتصادی جهان محسوب می‌شود.
