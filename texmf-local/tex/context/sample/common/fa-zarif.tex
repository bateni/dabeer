%% source: Farsi Wikipedia
%\externalfigure[zarif.jpg][left]
%
\placefigure[left,none]{}{\externalfigure[fa-zarif.jpg][width=4cm]}
%
محمدجواد ظریف (‎۱۷ دی ۱۳۳۸ در تهران) دیپلمات ایرانی و وزیر امور خارجه
کنونی جمهوری اسلامی ایران، عضو شورای عالی جمعیت هلال احمر جمهوری
اسلامی ایران و استاد میهمان دانشکدهٔ حقوق و علوم سیاسی دانشگاه تهران
است. ظریف به مدت ۵ سال (از ۱۴ مرداد سال ۱۳۸۱ تا ۵ تیر سال ۱۳۸۶)
سفیر و نمایندهٔ دائم جمهوری اسلامی ایران در سازمان ملل متحد
بوده است. او خود را عضو هیچ‌یک از احزاب سیاسی ایران نمی‌داند. وی
رکوردهای مختلفی را در جمهوری اسلامی به نام خود ثبت کرده است: از انتصاب
اولین سخنگوی زن وزارت امور خارجه تا اولین سفیر زن جمهوری اسلامی و بیش
از ۵۰ بار دیدار با وزیر خارجه آمریکا و اولین دیدار و مصافحه با
رئیس‌جمهور آن کشور. ظریف همچنین بیش از سایر وزرای دیگر در تاریخ
انقلاب اسلامی در صحن مجلس و در کمیسیون امنیت ملی حاضر شده است و
دربارهٔ مسائل گوناگون به نمایندگان پاسخ داده است  ظریف در سال ۲۰۱۵
از سوی موسسه انگلیسی «گلوبال ریسک اینسایتس» به عنوان ریسک‌پذیرترین
سیاستمدار جهان انتخاب شد.
