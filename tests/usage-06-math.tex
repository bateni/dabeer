\usemodule[dabeer]

%% This is only for debugging and helps us understand better
%% what CONTEXT is doing with math formulas.  Final document
%% is supposed to turn the option off.
\tracemathtrue

\definemathframed[graymath]
[
  frame=off,
  location=mathematics,
  background=color,
  backgroundcolor=lightgray,
  backgroundoffset=2pt
  ]

%% Option two looks better for Persian digits.
%% Beware that automatic translation of Latin digits to Persian
%% is not performed.
\setconstant\vulgarfractionmethod\plustwo

%% some preliminary theorem/proof environments.
%% https://mailman.ntg.nl/pipermail/ntg-context/2014/079923.html
%% TODO: define a smart proof symbol
\defineenumeration[theorem][text=قضیه ,alternative=serried,width=fit,numberconversion=persiandecimals]
\defineenumeration[lemma][theorem][text=لم ,]
\defineenumeration[proof][number=no,text=اثبات.,alternative=serried,width=fit,headstyle=italic,closesymbol=\mathematics{\square}] %right=.,stopper=. did not work

\starttext

حروفچینی متون ریاضی همواره از نِقاط قوت \TEX\ بوده است.
از آن‌جایی که \CONTEXT\ نیز در این زمینه تقریباً با \TEX\ یکسان است
نباید تفاوت چشمگیری وجود داشته باشد.
با این حال، مهم‌ترین مسئله استفاده از ارقام فارسی در فرمول‌های ریاضی است.

به عنوان مثال، می‌دانیم که مجموع $n$ عدد صحیح نخست برابر است با $n\frac{n+1}{2}$.
به عبارت دیگر،
%
\startformula
\sum_{i=1}^n i = n\frac{n+1}{2}.
\stopformula

در جهان امروز، از کودک دبستانی گرفته تا افراد میان‌سال، کیست که فرمول معروف
\eqref[eq:mc2]
را نشنیده باشد و آن را منسوب به آلبرت اینشتین نداند؟
با این وجود، تعداد آنانی که درکی از این رابطه ریاضی دارند اندک است.

\placeformula[eq:mc2]
\startformula
E = mc^2.
\stopformula

در \CONTEXT\ استفاده از قاب‌ها بسیار ساده است.
مثلاً قاب موجود در فرمول زیر به‌سادگی نشان می‌دهد که 
 $\log(1+x) = x + O(x^2)$.

\startformula
\log(1+x) = \graymath{x- \displaystyle{x^2\over2}} + {x^3 \over 3} + \cdots
\stopformula



در ادامه به یک مثال پیچیده‌تر نگاهی می‌اندازیم.

%% We need to use Persian digits inside \vfrac if we want them.
%% The conversion is done automatically.
\placeformula
 \startformula
 \startalign
  \NC c^2 \EQ a^2 + b^2  \NR
  \NC c \EQ \left(a^2 + b^2\right)^{\vfrac{۱}{۲}} \NR
  \NC a^2 + b^2 \EQ c^2  \NR
  \NC d^2 \EQ e^2        \NR
 \stopalign
\stopformula

با استفاده از دستور \type{\mbox} می‌توان در محیط ریاضی از متن فارسی بهره برد.
\startformula
E = \{ x | \mbox{$x$ زوج است } \}.
\stopformula

یکی از زیباترین روابط ریاضی در ادامه می‌آید.
این رابطه پنج ثابت مهم ریاضیات را به یکدیگر می‌پیوندد.

\startformula
e^{i\pi} + 1 = 0.
\stopformula

برای حروفچینی عبارات چندخطی ریاضی بهتر است از \type{\startalign} بهره بگیریم.
صورت کلی برنامه خطی به شکل زیر است.

\placeformula
%\placesubformula  % does not work properly.  acts with normal formula numbering.
\startformula
\startalign[n=5,align={left,right,left,middle,right}]
\NC \min   \NC  \sum_i c_i x_i    \NC           \NC \quad \NC \NR[lp:0]
\NC \mbox{subject to }   \NC  \sum_i a_{ij} x_i \NC \geq b_j \NC \NC \forall j \NR %[+][lp:1]
\NC                      \NC  x_i               \NC \geq 0   \NC \NC \forall i. \NR %[+][lp:2]
\stopalign
\stopformula

همان طور که از فلسفه \CONTEXT\ برمی‌آید،
دستور \type{\definemathalignment}
امکان تعریف عملگری برای حروفچینی برنامه‌های خطی فراهم می‌نماید.


\theorem
مجموعه اعداد صحیح نسبت به عمل جمع بسته است.

\proof
یک عدد صحیح $x$ را در نظر بگیرید.
واقعاً مدعای فوق نیازی به استدلال ندارد و اظهر من الشمس است!

\lemma
نسبت محیط دایره به قطر آن برابر با عدد $\pi$ است.

\startlemma
\input fa-norooz
\stoplemma

\stoptext
