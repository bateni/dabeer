
\usemodule[dabeer]
\starttext


\index{آبا}
\index{اب}
سلام


%\index{Google}
\index{گوگل}
\index{سلام}
\index{شرکت}
\index{إداره}
\index{أداره} %% wrong word
\index{اداره}
\index{أدیم}
\index{ءداره} %% wrong word
\index{آداره} %% wrong word
\index{آدینه}
\index{آبیاری}


اهدافی که انتظار داریم از طریق نمایه‌زنی فارسی تأمین شود
به این شرح است.
\startitemize
\item
عنوان بخش مورد نظر \quote{نمایه} باشد.
\item
امکان حذف یا گذاشتن شماره بخش/فصل برای آن وجود داشته باشد.
\item
در حالت چندستونه، ستون‌های نمایه از راست به چپ باشند.
\item
فواصل زائد بین یک حرف اصلی و اولین مدخل ذیل آن حذف شوند.
\item
ترتیب حروف در نمایه باید بر اساس الفبای فارسی باشد.
\item
حروف اصلی نظیر ك و ي بایستی کاملاًً مانند حروف مشابه فارسی‌شان عمل کنند.
٪٪ یعنی چه؟
\item
حرکات و تشدید و \dots\ در
{\bf حالت ساده}
 نباید تغییری در ترتیب ایجاد نمایند.
\item
عنوان بخش مربوط به ه‍ و الف صحیح باشد.
\item
از ویرگول فارسی استفاده شود.
\item
توضیحات اضافه نظیر \quotation{رجوع شود} با رعایت شکل‌دهی حروف فارسی چیده شود.
\item
تکرار واژگانی نظیر \quote{آب} و \quote{اب} ایرادی برای مرتب‌سازی ایجاد ننماید.
\item
ترتیب عبارت‌هایی نظیر
\quote{سرعت}،
\quote{سرعتی}،
\quote{سرعت عمل} و
\quote{سرعت‌سنج}
درست باشد.
٪٪ ترتیب درست چیست؟
\item
مدخل انگلیسی در نمایه فارسی درست در سمت راست عدد صفحه جای بگیرد.
\item
تنظیمات مناسب برای داشتن نمایه‌ای انگلیسی هم فراهم باشد.
\stopitemize



\page
\index{آب}
\index{اب}
\index{آب}
\index{گوگل}
\index{شرکت}
\index{آبادی}
\index{بودا}
\index{تابستان}
\index{ثواب}
\index{پالتو}
\index{ابراهیم}
\index{عبارت}
\index{قالی}
\index{فالوده}
\index{فیل}
\index{فهم}
\index{فوتبال}
\index{رام}
\index{زار}
\index{ناز}
\index{ژانر}
\index{غبار}
\index{سرعت}
\index{سرعت‌سنج}
\index{سرعت عمل}
\index{سرعتی}
\index{چادر}
\index{جرأت}
\index{جرایم}
\index{جرائم}
\index{دستور}
\index{صولت}
\index{ظلم}
\index{طلب}
\index{قائم}
\index{کدورت}
\index{كائن} % arabic kaf
\index{حوله}
\index{خورش}
\index{گلیم}
\index{لبو}
\index{لب}
\index{لبوفروش}
\index{مدیریت}
\index{مدیریت+مدیر}
\index{نان}
\index{نان+نان‌آوری}
\index{ورّاج}
\index{وراثت}
\index{وروجک}
\index{هلو}
\index{هلال}
\index[تک]{\TEX}
\index[کانتکست]{\CONTEXT}
\seeindex[کانتکست]{\CONTEXT}{\TEX}
\index{تک}
\index{کانتسکت}
\seeindex{فیلم}{ژانر}
\index{یراق}
\index{يتيم} % arabic yeh
\index{ذرت}
سلام
\page
\index{سلام}
\index{گوگل}
\index{اب}
\index{آب}



%\page
\completeindex

\stoptext
