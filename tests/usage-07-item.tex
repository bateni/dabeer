\usemodule[dabeer]

\starttext

\startitemize[ع]
\item مورد نخست
\item مورد بعدی
\item و این یکی \dots
  \startitemize[ا]
  \item در ادامه
  \item و هنوز
  \stopitemize
\item دیگر بس است!
\stopitemize

%% start/stopitem are needed for the row-first column mode.
\startbuffer[itemlist]
\startitem فروردین \stopitem
\startitem اردیبهشت \stopitem
\startitem خرداد \stopitem
\startitem تیر \stopitem
\startitem مرداد \stopitem
\startitem شهریور \stopitem
\startitem مهر \stopitem
\startitem آبان \stopitem
\startitem آذر \stopitem
\startitem دی \stopitem
\startitem بهمن \stopitem
\startitem اسفند \stopitem
\stopbuffer

اینک چند مثال از فقرات چندستونه می‌بینیم.
کافی است از کلید \type{columns} در تنظیمات فقرات استفاده کرده
و تعداد ستون‌ها را به حروف بنویسید.
در نمونه نخست برای صرفه‌جویی در فضا از کلید \type{packed} نیز بهره برده‌ایم.

\startitemize[ع,columns,two,packed]
\getbuffer[itemlist]
\stopitemize

\startitemize[ع,columns,three]
\getbuffer[itemlist]
\stopitemize

\startitemize[ع,columns,four]
\getbuffer[itemlist]
\stopitemize

\startitemize[ع,columns,five]
\getbuffer[itemlist]
\stopitemize

دقت کنید که پایان صفحه ممکن است شمایل غیرمنتظره‌ای در این گونه فقرات
پدید آورد.  با این حال، شماره‌ها منطقی است و تمام اعداد
اختصاص‌داده‌شده در صفحه اول پیش از اعداد صفحه دوم خواهند بود.

افزون بر این، گاهی تعداد ستون‌ها با آن چه شما انتظار دارید متفاوت از
آب در می‌آید.  مثلاً در نمونه زیر، ما درخواست پنج ستون را داده‌ایم،
اما تعداد فقرات به گونه‌ای است که دو سطر پنج‌ستونه کفاف همه را
نمی‌دهد؛ و از سوی دیگر، پس از تعبیه نمودن ۳ سطر، ۴ ستون نخست همه فقرات
را پوشش داده و ستون پنجم خالی خواهد ماند.

\startitemize[ع,columns,five]
\getbuffer[itemlist]
\stopitemize


حال اگر به جای \type{columns} از کلید \type{horizontal} استفاده
نماییم، شماره‌گذاری فقرات اولویت را به سطرها می‌دهد.  این نوع
شماره‌گذاری مناسب پرسش‌های چندگزینه‌ای است.  با این حال، توجه داشته
باشید که در این‌جا فابلیت شکستن خطوط در میان یک فقره وجود ندارد.

\startitemize[ع,horizontal,four]
\getbuffer[itemlist]
\stopitemize


\stoptext

