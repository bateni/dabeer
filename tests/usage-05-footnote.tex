\usemodule[dabeer]
\setuppapersize[A5][A5]

%% TODO: None of these work.  The difference between 1st and 2nd lines
%% is that one uses normal unicode chars and the other uses the presentation
%% forms.
%\hyphenation{پر-بازدید-کننده‌-ترین}
%\hyphenation{ﭘﺮ-ﺑﺎزدﯾﺪ-ﮐﻨﻨﺪه-ﺗﺮﯾﻦ}
%\hyphenation{پر\-بازدید\-کننده\-ترین}

\starttext

در این‌جا به دنبال اهداف زیر هستیم.
% packed is almost the same as nowhite but places blank before and after the group.
\startitemize[nowhite]
\item متن درون زیرنویس بایستی با الگوریتم دوجهتی پردازش شود.
\item زیرنویس‌هایی که با استفاده از \type{\footnote} درج می‌شوند از
  جهت متن حول خود برای تعیین سمت زیرنویس بهره می‌گیرند: اگر متن پیش از
  آن‌ها راست‌به‌چپ بوده است، زیرنویس نیز در سمت راست صفحه جای می‌گیرد؛
  و در غیر این صورت شاهد زیرنویس در سمت چپ صفحه خواهیم بود.
\item زیرنویس‌های سمت راست با ارقام فارسی حروفچینی شوند و زیرنویس‌های
  سمت چپ با ارقام لاتین.
\item جهت نخستین زیرنویس صفحه مکان خط جداکننده زیرنویس‌ها را تعیین نماید.
\item شماره زیرنویس در متن پیرو جهت متن پیش از آن باشد.
\item دو دستور \type{\LTRfootnote} و \type{\RTLfootnote} مشابه دستور
  اصلی بوده، اما برای تعیین جهت متن زیرنویس از پیش‌فرض استفاده
  نمی‌نماید.
\stopitemize

\blank

\startbuffer[farsifoot]
این نمونه یک زیرنویس فارسی است.
\stopbuffer

\startbuffer[farsinumfoot]
سال ۲۰۱۶ کبیسه است.
\stopbuffer

\startbuffer[englishfoot]
This is an English footnote.
\stopbuffer

کار را با یک زیرنویس%
\footnote{\getbuffer[farsifoot]}
فارسی آغاز می‌کنیم و یک زیرنویس%
\footnote{\getbuffer[englishfoot]}
انگلیسی نیز می‌افزاییم.
این زیرنویس%
\footnote{\getbuffer[farsinumfoot]}
شامل عدد می‌شود.

\startLTR 
Next we have an English paragraph with a Farsi
footnote\footnote{\getbuffer[farsifoot]} and an
English\footnote{\getbuffer[englishfoot]} one.  
\stopLTR

\input fa-google

\startLTR 

Next we have an English paragraph with an English
footnote\footnote{\getbuffer[englishfoot]} and then a
Farsi\footnote{\getbuffer[farsifoot]} one.  Let us then force an RTL
footnote\RTLfootnote{\getbuffer[farsifoot]} inside this LTR paragraph.

\stopLTR

و در این متن راست‌به‌چپ یک زیرنویس%
\LTRfootnote{\getbuffer[englishfoot]} چپ‌به‌راست درج می‌نماییم.  در
ادامه به اعداد زیرنویس‌ها دقت کنید: اعداد بزرگ‌تر از ۱۰ به صورت صحیح
نوشته شده‌اند.

\dorecurse{3}{سلام\footnote{\getbuffer[farsifoot]} و } \dots

\startLTR
\dorecurse{3}{Hello\footnote{\getbuffer[englishfoot]} and } \dots
\stopLTR

\page

خط جداساز زیرنویس‌ها%
\LTRfootnote{English text.}
 باید در سمت چپ قرار گیرد.

\page

\startLTR

The footnote\RTLfootnote{متن فارسی.} should come after a right-aligned footnote rule.

\stopLTR

\stoptext

