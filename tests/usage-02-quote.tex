\setuppapersize[A5]
\setuplayout[width=middle,margin=1cm]

\usemodule[dabeer]
\setupbodyfont[dabeerformal]

\defineblank[ExtractDistance][3pt]
\definestartstop[BlockQuote]
                [
                  before={\blank[ExtractDistance]
                    \setupnarrower[left=1.5pc,right=1.5pc]
                    \startnarrower[left,right]
                    \noindent},
                  after={\stopnarrower
                    \blank[ExtractDistance]
                    \indenting[next]}]

\starttext

\showframe


\quotation{نهج‌البلاغه}
کتاب ارزشمندی است که سیدرضی برخی از بلیغ‌ترین سخنان و نوشتارهای
منسوب به امیر مؤمنان علی‌بن‌ابی‌طالب را در آن گرد آورده است.
امام علی علیه‌السلام در کلمه قصار ۱۹۶
می‌فرمایند:
\quotation{آنچه از مالَت رفت و تو را پند آموخت، از دستت نشُد و نسوخت.}

در خطبه ۱۷۷ این کتاب می‌خوانیم:

\startBlockQuote
خوشا کسی که پرداختن به عیب خویش وی را از عیب دیگران بازدارد. خوشا کسی
که در خانه نشیند و قوت خود خورد، و به فرمانبرداری پروردگار روی آرد، و
بر گناهِ خود بگرید تا سرگرمِ کار خویش باشد و مردم را از گزند خود آسوده
گذارد.
\stopBlockQuote

{\bf توضیح اول}: 
برگردان‌ها از ویکی‌گفتاورد گرفته شده است.

{\bf توضیح دوم}:
قطع A5 برای صفحه در نظر گرفته شده است و {\it متن} یا {\it بدنه}\LTRfootnote{body} را در میانه آن قرار داده‌ایم.
همچنین، پهنای {\it حاشیه}\LTRfootnote{margin} را نیز یک سانتی‌متر تعیین کرده‌ایم.
برای نمایش قاب دور صفحه از دستور 
\type{\showframe} بهره برده‌ایم.
اگر دستور \type{\showlayout} را به کار می‌گرفتیم، علاوه بر این قاب، مقدار پارامترهای دخیل در قاب‌بندی و صفحه‌بندی نیز فهرست می‌شد.
\stoptext

